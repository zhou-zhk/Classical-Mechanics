\documentclass[11pt]{article}
\usepackage[utf8]{inputenc}	% Para caracteres en español
\usepackage{amsmath,amsthm,amsfonts,amssymb,amscd}
\usepackage{multirow,booktabs}
\usepackage[table]{xcolor}
\usepackage{fullpage}
\usepackage{lastpage}
\usepackage{enumitem}
\usepackage{fancyhdr}
\usepackage{mathrsfs}
\usepackage{wrapfig}
\usepackage{setspace}
\usepackage{calc}
\usepackage{multicol}
\usepackage{cancel}
\usepackage[retainorgcmds]{IEEEtrantools}
\usepackage[margin=3cm]{geometry}
\usepackage{amsmath}
\newlength{\tabcont}
\setlength{\parindent}{0.0in}
\setlength{\parskip}{0.05in}
\usepackage{empheq}
\usepackage{framed}
\usepackage[most]{tcolorbox}
\usepackage{xcolor}
\usepackage{graphicx} % Required for inserting images
\usepackage{extarrows} % long equal symbol on which words can be set upon it
\colorlet{shadecolor}{orange!15}
\parindent 0in
\parskip 12pt
\geometry{margin=1in, headsep=0.25in}
\theoremstyle{definition}
\newtheorem{defn}{Definition}
\newtheorem{reg}{Rule}
\newtheorem{exer}{Exercise}
\newtheorem{note}{Note}
\begin{document}
% \setcounter{section}{8}
\title{Lagrangian Mechanics}
\date{\today}

\thispagestyle{empty}

\begin{center}
{\LARGE \bf Lagrangian Mechanics}\\
{\large Theoretical Mechanics}\\
% Fall 2025
\end{center}

\section{Variational principles}
\subsection{Calculus of Variations}  % using the calculus of variations to study the mechanical problems in physics
The \textbf{calculus of variations}, is concerned with the extremals of functions whose domain is an infinite-dimensional space: the space of curves. Such functions are called \textbf{functionals}. 
\subsubsection{Variations and Extremals}
Definitions of \textbf{differentiable} and \textbf{differential}: the functional $\Phi$ is \textit{differentiable} if $\Phi (\gamma +h) - \Phi(\gamma) = F+R$ where $F$ depends linearly on $h$. The linear part $F(h)$ is the \textit{differential} of $\Phi$.

Example: A curve $\gamma$ in $(t,x)$-plane: $\gamma = \{ (t,x): x=x(t), t_0 \leq t \leq t_1 \} $. 

\subsubsection{The Euler-Lagrange Equation}
\textbf{Definition}: the equation \begin{equation}
    \frac{\mathrm{d}}{\mathrm{d}t} \left( \frac{\partial L}{\partial \dot{x}} \right) - \frac{\partial L}{\partial x} =0
\end{equation} is the \textbf{Euler-Lagrange equation} of the functional \begin{equation}
    \Phi = \int_{t_0}^{t_1} L(x,\dot{x}, t) \mathrm{d}t. 
\end{equation}
\textbf{Theorem}: the curve $\gamma$ is an extremal of the functional $\Phi(\gamma) = \int_{t_0}^{t_1} L(x,\dot{x}, t) \mathrm{d}t$ on the space of curves joining $(t_0,x_0)$ and $(t_1, x_1)$ if and only if the Euler-Lagrange equation is satisfied along $\gamma$. 

The condition for a curve y to be an extremal of a functional does not depend on the choice of coordinate system.



\newpage
\subsection{Legendre transformation}
\textbf{Definition}: Let $y=f(x)$ be a convex function which means $f''(x)>0$. And the \textbf{Legendre transformation} of the function $f$ is a new function $g$ of a new variable $p$, which is constructed in such a method: Draw the graph of $f$ in $x-y$ plane, and consider the line $y=px$ in which $p$ is a given number. For each $p$ the function $F(p,x)=px-f(x)$ has a maximum value with respect to $x$ at the point $x(p)$. We define $g(p)=F(p,x(p))$. 

\begin{figure}[h]
    \centering
    \includegraphics[width=0.5\linewidth]{Fig 1-3-1.png}
    \caption{Legendre transformation}
    \label{Fig 1-3-1}
\end{figure}

The point $x(p)$ is defined by the extremal condition $\partial F / \partial x=0$, and $f'(x)=p$. The point $x(p)$ is unique for convex function $f$. 

\subsection{Hamilton's equations}
\subsubsection{Equivalence of Lagrange's and Hamilton's equations}
Assuming Lagrangian $L$ is convex with respect to $\dot{q}$, and the system of Lagrange's equation $\dot{p}=\partial L / \partial q$. 

\textbf{Theorem}: The system of Lagrange's equations is equivalent to the system of $2n$ first-order equations (Hamilton's equations)
\begin{align}
    & \dot{\mathrm{p}} = -\frac{\partial H}{\partial \mathrm{q}} \\
    & \dot{\mathrm{q}} = \frac{\partial H}{\partial \mathrm{p}}, 
\end{align}
where $H(\mathrm{p}, \mathrm{q}, t) = \mathrm{p}\mathrm{q} - L (\mathrm{q}, \dot{\mathrm{q}}, t)$ is the Legendre transform of the lagrangian function, which is viewed as a function of $\dot{\mathrm{q}}$. 

Proof of the theorem: The Legendre transform of $L$ with respect to $\dot{q}$ is $H(p) = p \dot{q} - L(\dot{q})$, since $\frac{\partial H}{\partial \dot{q}} = 0$ in the Legendre transform, we get $p=\frac{\partial L}{\partial \dot{q}}$, so it depends on $q$ and $t$. The total differential of $hamiltonian \; H$ is \begin{equation}
    \mathrm{d}H = \frac{\partial H}{\partial \mathbf{p}} \mathrm{d}\mathbf{p} + \frac{\partial H}{\partial \mathbf{q}} \mathrm{d}\mathbf{q} +\frac{\partial H}{\partial t} \mathrm{d}t \xlongequal[p=\partial L / {\partial\dot{q}}]{H=p\dot{q}-L} \mathbf{\dot{q}} \mathrm{d}\mathbf{p} - \frac{\partial L}{\partial \mathbf{q}} \mathrm{d}\mathbf{q} -\frac{\partial L}{\partial t} \mathrm{d}t. 
\end{equation} So we get \begin{equation}
    \mathbf{\dot{q}} = \frac{\partial H}{\partial \mathbf{p}}, \;\; \frac{\partial H}{\partial \mathbf{q}} = - \frac{\partial L}{\partial \mathbf{q}}, \;\; \frac{\partial H}{\partial t}=-\frac{\partial L}{\partial t}. 
\end{equation} Since for Lagrangian $L$: $\frac{\mathrm{d}}{\mathrm{d}t} \left( \frac{\partial L}{\partial \dot{q}} \right) - \frac{\partial L}{\partial q} =0$, we have $\frac{\partial H}{\partial \mathbf{q}} = - \frac{\partial L}{\partial \mathbf{q}}=-\frac{\mathrm{d}}{\mathrm{d}t} \frac{\partial L}{\partial \dot{\mathbf{q}}} = -\frac{\mathrm{d}}{\mathrm{d}t}\mathbf{p} = -\mathbf{\dot{p}}$ so Hamilton's equations are obtained. \\ If $q(t)$ satisfies Lagrange's equations, $(p(t), q(t))$ also satisfies Hamilton's equations, and vice versa. Therefore \textbf{the systems of Lagrange and Hamilton are equivalent}.


\subsubsection{Hamilton's function and energy} 
Lagrangian has the usual form $L=T-U$, where $T$ is a quadratic form with respect to $\mathbf{\dot{q}}$
\begin{equation}
    T = \frac{1}{2} a_{ij} \dot{q}_{i}\dot{q}_{j}, \;{\rm where} \;a_{ij}= a_{ij}(\mathbf{q},t) \;{\mathrm{and}}\; U=U(\mathbf{q})
\end{equation}
\textbf{Theorem}: The Hamiltonian $H$ is the total energy $H=T+U$. 

Lemma: The values of a quadratic form $f(x)$ and of its Legendre transform $g(p)$ coincide at corresponding points: $f(x)=g(p)$. 

Proof of the lemma: for a quadratic form $f(x)=\frac{1}{2}\sum_{i,j} a_{ij} x_i x_j$ which is a homogeneous function of degree 2, it satisfies $f(\lambda x) = \lambda^2 f(x)$ and $x\cdot \nabla f(x)=2f(x)$ (for a homogeneous function $f(x)$ of degree $n$, $\frac{\mathrm{d}}{\mathrm{d}\lambda} f(\lambda x) = n\lambda ^ {n-1} f(x)$, by the chain rule $\frac{\mathrm{d}}{\mathrm{d}\lambda} f(\lambda x) = \frac{\mathrm{d}}{\mathrm{d} (\lambda x)} f(\lambda x)\cdot \frac{\mathrm{d} (\lambda x)}{\mathrm{d} \lambda} = \nabla f(\lambda x)\cdot x$, so $\nabla f(\lambda x)\cdot x = n\lambda ^ {n-1} f(x)$, letting $\lambda =1$ and we get $x\cdot \nabla f(x)= nf(x)$, which is the \textbf{Euler’s Theorem} on homogeneous functions), so if $f$ is strictly convex, $g(p)=px-f(x)=\frac{\partial f(x)}{\partial x}x-f(x)=2f(x)-f(x)=f(x)$. 

Proof of the theorem: $H = p\dot{q}-L = \frac{\partial L}{\partial \dot{q}} \dot{q}-L$, since $L=T(q,\dot{q},t)-U(q)$, applying Euler’s theorem and we get $H = \frac{\partial T}{\partial \dot{q}} \dot{q}-(T-U)=2T-(T-U)=T+U$. 

\textbf{Corollary} $\frac{\mathrm{d}H}{\mathrm{d}t} = \frac{\partial H}{\partial t}$: If a system whose $hamiltonian$ function satisfies $\frac{\partial H}{\partial t}=0$, which means does not depend explicitly on time, the laws of conservation holds: $H(\mathbf{p}(t), \mathbf{q}(t)=const$.

Proof of the corollary: \begin{equation}
    \frac{\mathrm{d}H}{\mathrm{d}t} = \frac{\partial H}{\partial p} \frac{\partial p}{\partial t} + \frac{\partial H}{\partial q}\frac{\partial q}{\partial t} + \frac{\partial H}{\partial t} = \frac{\partial H}{\partial p} \left(- \frac{\partial H}{\partial q} \right) + \frac{\partial H}{\partial q} \frac{\partial H}{\partial p} + \frac{\partial H}{\partial t}=\frac{\partial H}{\partial t}.
\end{equation}


\subsubsection{Cyclic coordinates}



\newpage
\subsection{Liouville's Theorem}
\textbf{Definition}: \textbf{Phase space} is the $2n$-dimensional space with coordinates $p_1 , \dots , p_n ; q_1, \dots , q_n$. \textbf{Phase flow} is the one-parameter group of transformations of phase space \begin{equation}
    g^t : (\mathbf{p}(0),\mathbf{q}(0)) \xrightarrow{} (\mathbf{p}(t),\mathbf{q}(t))
\end{equation}
where $\mathbf{p}(t)$ and $\mathbf{q}(t)$ are solutions of Hamilton's system of equations. 

\textbf{Theorem 1}: The phase flow preserves volume: for any region $D$, the volume $V$ satisfies $V(g^t D)=V(D)$. 


\subsubsection{Liouville's theorem and its proof}

\subsubsection{Poincare's recurrence theorem}


\newpage
\section{Lagrangian mechanics on manifolds}
\subsection{Holonomic constraints}
Let $\gamma$ be a smooth curve in the plane. For curvilinear coordinates $q_1$ on $\gamma$, and $q_2$ on a neighborhood of $\gamma$. 

considering the system with potential energy: \begin{equation}
    U_N = N q_{2}^2 + U_0 (q_1, q_2),
\end{equation} in which $N$ is the parameter tending to infinity. The evolution of the coordinate $q_1$ under a motion with these initial conditions... 


\subsection{Differentiable manifolds}

\subsection{Lagrangian dynamical systems}

\subsection{E. Noether's theorem}

\subsection{D'Alembert's principle}


\begin{comment}

\newpage
\subsection{Acceleration Without Rotation}
Consider an inertial reference frame (i.e not accelerating) which will be denoted S$_0$, and a accelerating reference frame, \textit{S} that has an acceleration of \textit{A}. 
\begin{note}
\textbf{Capital Letters refer to the accelerating reference frame \textit{S} while lowercase letters refer to the inertial reference frame S$_0$}
\end{note}
Picture a moving reference frame, \textit{S}, moving relative to S$_0$. Imagine in the the moving reference frame that a ball with mass, \textit{m} is being thrown. 
In order to consider the motion of the ball, the motion must be first considered in the inertial reference frame. 
\begin{equation}
F = m\ddot{r_0}
\end{equation}
Where r$_0$ is the ball's position relative to S$_0$. 

Now, by considering the motion of the ball in the accelerating frame, the ball position relative to \textit{S} is \textit{R}. (It's velocity is $\dot{R}$. 
Thus, relating \textit{R} to $r_0$, we have: 
\begin{equation}
\dot{r_0} = \dot{R} + V
\end{equation}
Newton's second law for the inertial reference frame by differentiate and multiplying by mass is:
\begin{equation}
F_{\text{inertial}} = -mA = -m\ddot{R}
\end{equation}
\subsection{The Tides}
\begin{shaded}
\textbf{The Tidal Force} \newline
\begin{equation}
F_{tide} = -GM_mm(\frac{\hat{d}}{d^2}-\frac{\hat{d_0}}{d_0^2})
\end{equation}
Where:
\begin{equation*}
\begin{split}
G = \text{Gravitational Constant} \\
d = \text{Object's Position Relative to Moon} \\
d_0 = \text{Earth's Center Relative to the moon}\\
M_m = \text{Mass of the moon}
\end{split}
\end{equation*}
\end{shaded}
\newpage
\subsection{The Angular Velocity Vector}
The rest of the notes and the chapter will over reference frames that are rotating with respect to the inertial reference frame, so angular velocity has to be used. 
\begin{defn}
\textbf{Euler's Theorem} - The most general motion of any body relative to a fixed point \textit{O} is a rotation about some axis through \textit{O} To specify this rotation about a given point O, we only have to give the direction of the axis and the rate of rotation, or angular velocity $\omega$. Because this has a magnitude and direction, it is an obvious choice to write this rotation vector as $\omega$, the angular velocity vector. That is:
\begin{equation}
\omega = \omega\textbf{u}
\end{equation}
Where \textbf{u} is the unit vector
\end{defn}
\begin{shaded}
\textbf{Vector Velocity}\newline
The velocity at any point, \textit{P} (position, \textit{r}) is given by:
\begin{equation}
v = \omega\  x \ r
\end{equation}
\end{shaded}
\subsection*{Addition of Angular Velocities}
One can add angular velocities just like linear velocities. If body 3 is rotating at angular velocity $\omega_{32}$ relative to frame 2, and frame 2 is rotating at angular velocity $\omega_{21}$ relative to frame 1, then body 3 is rotating relative to frame 1 at angular velocity: 
\begin{equation}
\omega_{31} = \omega_{32} + \omega_{21}
\end{equation}
\subsection{Time Derivatives in Rotating Frames}
If frame S has a angular velocity, $\Omega$ relative to S$_0$ then the time derivative of a single vector \textbf{Q} as seen in the two frames are related by:
\begin{equation}
(\frac{d\textbf{Q}}{dt})_{S_0} = (\frac{d\textbf{Q}}{dt})_{S} \ + \Omega \ x \ \textbf{Q}
\end{equation}
\subsection{Netwon's Second Law in a Rotating Frame}
A particle in an inertial reference frame, S$_0$ obeys Newton's second law as we are use to:
\begin{equation}
m\frac{d^2r}{dt^2} = F
\end{equation}
Using the results from equation 8, the time derivative for a rotating frame with reference to an inertial frame can be given by:
\begin{equation}
(\frac{dr}{dt})_{S_0} = (\frac{dr}{dt})_s \ + \Omega \ x \ r
\end{equation}
By differentiation, Newton's second law becomes:
\begin{equation}
m\ddot{r} = F + 2m\dot{r} \ x \ \Omega \ + m(\Omega \ x \ r) \ x \ \Omega
\end{equation}
Where \textit{F} is the sum of all forces in the inertial reference frame. 
\subsection{The Centrifugal Force}
This is an inertial force in a rotating reference frame 
\begin{equation}
F_{\text{cf}} = m(\Omega \ x \ r) \ x \ \Omega
\end{equation}
\subsubsection*{Free-Fall Acceleration (Non-Vertical Gravity)}
\begin{equation}
F_{\text{eff}} = F_{\text{grav}} + F_{\text{cf}} = mg_0 + m\Omega^2R\sin(\theta)\hat{\rho}
\end{equation}
The acceleration due to the Centrifugal force is simply 
\begin{equation}
\begin{split}
g = g_0 + \Omega^2R\sin(\theta)\hat{\rho} \\
g_{\text{rad}} = g_0 - \Omega^2R\sin^2(\theta)  \\
g_{\text{tan}} = \Omega^2R\sin(\theta)\cos(\theta)
\end{split}
\end{equation}
The angle between g and its radial direction is:
\begin{equation}
\alpha \approx \frac{g_{\text{tan}}}{g_{\text{rad}}} 
\end{equation}
The maxium value at ($\theta$ = 45):
\begin{equation}
\alpha_{\text{max}} =  \frac{\Omega^2R}{2g_0}
\end{equation}
\subsection{Coriolis Force}
The Coriolis Force is another inertial force in a rotating reference frame that an object experiences when it is moving. 
\begin{equation}
F_{\text{cor}} = 2m\dot{r} \ x \ \Omega = 2mv \ x \ \Omega
\end{equation}
The maximum acceleration, \textit{a} that the Coriolis force could produce acting by itself with \textit{v} perpendicular to $\Omega$ is:
\begin{equation}
a_{\text{max}} = 2v\Omega 
\end{equation}
\begin{shaded}
\textbf{Direction of the Coriolis Force} \newline
The Direction of the Coriolis force us always perpendicular to the velocity of the object (hence equation 17), and is given by the right hand rule. 
\end{shaded}
\newpage
\subsection{Free Fall and the Coriolis Force}
\begin{equation}
m\ddot{r} = mg_0 + F_{\text{cf}} + F_{\text{cor}} 
\end{equation}
\subsection{The Foucault Pendulum}
See chapter 9, Page 354. There is no need to recopy what is in the book here. 


\end{comment}


\end{document}
